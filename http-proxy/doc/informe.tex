\documentclass[a4paper,10pt]{article}

\usepackage[utf8]{inputenc}
\usepackage{t1enc}

\usepackage[utf8]{inputenc}
\usepackage{t1enc}
\usepackage[spanish]{babel}
\usepackage[pdftex,usenames,dvipsnames]{color}
\usepackage[pdftex]{graphicx}
\usepackage{enumerate}
\usepackage{amsmath}
\usepackage{amsfonts}
\usepackage{amssymb}
\usepackage[table]{xcolor}
\usepackage[small,bf]{caption}
\usepackage{float}
\usepackage{subfig}
\usepackage{listings}
\usepackage{bm}
\usepackage{times}

\begin{document}

\renewcommand{\lstlistingname}{C\'odigo Fuente}
\lstloadlanguages{Octave} 
\lstdefinelanguage{MyOctave}[]{Octave}{
        deletekeywords={beta,det},
        morekeywords={repmat}
} 
\lstset{
        language=MyOctave,
        stringstyle=\ttfamily,
        showstringspaces = false,
        basicstyle=\footnotesize\ttfamily,
        commentstyle=\color{gray},
        keywordstyle=\bfseries,
        numbers=left,
        numberstyle=\ttfamily\footnotesize,
        stepnumber=1,                   
        framexleftmargin=0.20cm,
        numbersep=0.37cm,              
        backgroundcolor=\color{white},
        showspaces=false,
        showtabs=false,
        frame=l,
        tabsize=4,
        captionpos=b,               
        breaklines=true,             
        breakatwhitespace=false,      
        mathescape=true
}

%%%%%%%%%%%%%%%%%%%%%%%%%%%%%%%%%%
%%%%%%%% begin TITLE PAGE %%%%%%%%
%%%%%%%%%%%%%%%%%%%%%%%%%%%%%%%%%%
\begin{titlepage}
        \vfill
        \thispagestyle{empty}
        \begin{center}
                \includegraphics{./images/itba.jpg}
                \vfill
                \Huge{Trabajo Pr\'actico Especial - Pre-entrega}\\
                \vspace{1cm}
                \Huge{Protocolos de Comunicaci\'on}\\
        \end{center}
        \vfill
        \large{
        \begin{tabular}{lcr}
                Crespo, Alvaro && 50758 \\
                Susnisky, Dar\'io && 50592\\
                Wassington, Axel && 50124\\
        \end{tabular}
}
        \vspace{2cm}
        \begin{center}
                \large{9 de mayo de 2012}\\
        \end{center}
\end{titlepage}
\newpage

\setcounter{page}{1}

\tableofcontents

\section{RFCs relevantes para el desarrollo de este trabajo pr\'actico}
\begin{itemize}
 \item rfc2616 --- http 1.1
 \item rfc2119 --- keywords to indicate requirement levels
 \item rfc822 ---- Backus-Naur Form (BNF)
 \item rfc2047 --- Message Header Extensions non ASCII-text (???)
 \item rfc2396 --- Uniform Resource Identifiers (URI): Generic Syntax

 \item rfcs 1950,51,52 --- gzip enconding and deflate enconding
 \item rfc1766 --- Tags for the Identification of Languages
 \item rfc1738 ---Uniform Resource Locators (URL)
\end{itemize}


\newpage
\section{Protocolos desarrollados}

\subsection{Protocolo para acceder al monitoreo del proxy}

Puerto X reservado para acceder al monitoreo del proxy.

User: username.
Pass: password.

En caso de autenticarse correctamente, el usuario podr\'a acceder a la información de monitoreo.

Bytes-transferred: cantidad total de bytes transferidos.
Bytes-transferred-clients: bytes transferidos entre clientes y el proxy.
Bytes-transferred-servers: bytes transferidos entre el proxy y los origin servers.
Total-filters-applied:
All-access:
Ip-filter:
URL-filter:
Media-type-filter:
Size-filter:
L33t-transformations:
Images-transformations:
Connections-open: cantidad de conexiones abiertas (clientes y origin server)


200 OK

401 UNAUTHORIZED

\newpage
\subsection{Protocolo para acceder a la configuración del proxy}

Puerto Y reservado para acceder a la configutaci\'on del proxy.

config-filter 1 [block | unblock]
config-filter 2 [block | unblock] [ip-adress or mask]
config-filter 3 [block | unblock] url-regexp
config-filter 4 [block | unblock] media-type
config-filter 5 [block | unblock] size-in-bytes
config-filter 6 [enable | disable]
config-filter 7 [enable | disable]


200 OK

400 BAD REQUEST

500 INTERNAL ERROR

\newpage
\section{Problemas y dificultades encontrados}

concurrencia
selector 1 thread  varios thread --> sincronizacion hiper heavy!

\newpage
\section{Casos de prueba}



\end{document}
