\documentclass[a4paper,10pt]{article}

\usepackage[utf8]{inputenc}
\usepackage{t1enc}

\usepackage[utf8]{inputenc}
\usepackage{t1enc}
\usepackage[spanish]{babel}
\usepackage[pdftex,usenames,dvipsnames]{color}
\usepackage[pdftex]{graphicx}
\usepackage{enumerate}
\usepackage{amsmath}
\usepackage{amsfonts}
\usepackage{amssymb}
\usepackage[table]{xcolor}
\usepackage[small,bf]{caption}
\usepackage{float}
\usepackage{subfig}
\usepackage{listings}
\usepackage{bm}
\usepackage{times}

\begin{document}

\renewcommand{\lstlistingname}{C\'odigo Fuente}
\lstloadlanguages{Octave} 
\lstdefinelanguage{MyOctave}[]{Octave}{
        deletekeywords={beta,det},
        morekeywords={repmat}
} 
\lstset{
        language=MyOctave,
        stringstyle=\ttfamily,
        showstringspaces = false,
        basicstyle=\footnotesize\ttfamily,
        commentstyle=\color{gray},
        keywordstyle=\bfseries,
        numbers=left,
        numberstyle=\ttfamily\footnotesize,
        stepnumber=1,                   
        framexleftmargin=0.20cm,
        numbersep=0.37cm,              
        backgroundcolor=\color{white},
        showspaces=false,
        showtabs=false,
        frame=l,
        tabsize=4,
        captionpos=b,               
        breaklines=true,             
        breakatwhitespace=false,      
        mathescape=true
}

%%%%%%%%%%%%%%%%%%%%%%%%%%%%%%%%%%
%%%%%%%% begin TITLE PAGE %%%%%%%%
%%%%%%%%%%%%%%%%%%%%%%%%%%%%%%%%%%
\begin{titlepage}
        \vfill
        \thispagestyle{empty}
        \begin{center}
                \includegraphics{./images/itba.jpg}
                \vfill
                \Huge{Trabajo Pr\'actico Especial - Pre-entrega}\\
                \vspace{1cm}
                \Huge{Protocolos de Comunicaci\'on}\\
        \end{center}
        \vfill
        \large{
        \begin{tabular}{lcr}
                Crespo, Alvaro && 50758 \\
                Susnisky, Dar\'io && 50592\\
                Wassington, Axel && 50124\\
        \end{tabular}
}
        \vspace{2cm}
        \begin{center}
                \large{9 de mayo de 2012}\\
        \end{center}
\end{titlepage}
\newpage

\setcounter{page}{1}

\tableofcontents

\section{RFCs relevantes para el desarrollo de este trabajo pr\'actico}
\begin{itemize}
 \item rfc2616 --- http 1.1
 \item rfc2119 --- keywords to indicate requirement levels
 \item rfc822 ---- Backus-Naur Form (BNF)
 \item rfc2047 --- Message Header Extensions non ASCII-text (???)
 \item rfc2396 --- Uniform Resource Identifiers (URI): Generic Syntax

 \item rfcs 1950,51,52 --- gzip enconding and deflate enconding
 \item rfc1766 --- Tags for the Identification of Languages
 \item rfc1738 ---Uniform Resource Locators (URL)
\end{itemize}


\newpage
\section{Protocolos desarrollados}

\subsection{Introducci\'on}

A continuación se presentar las especificaciones para dar un breve entendimiento 
de los protocolos para acceder al monitoreo del proxy y para acceder a la configuraci\'on del mismo.

\subsubsection{Proposito}

Tanto el protocolo para acceder al monitoreo del proxy como el protocolo para acceder a la 
configuración de este son protocolos que funcionan a nivel aplicación para sistemas de informaci\'on.

El protocolo para configurar al proxy pretende enviar los datos crudos para lograr su objetivo. En cambio, 
el protocolo para el monitero requiere un nivel de autenticaci\'on para luego recibir la informaci\'on 
deseada. 

Ninguno de los dos protocolos posee configuración extendida dando uso especifico tanto para el monitoreo 
como para la configuración particular de un proxy.

\subsubsection{Requerimientos}

Las palabras DEBE, NO DEBE,REQUERIDO, HARA, NO HARA, DEBERIA, NO DEBERIA, 
RECOMENDADO, PUEDE y OPCIONAL se deben interpretar tal y como describe el 
RFC2119 (http://www.rfc-es.org/rfc/rfc2119-es.txt).

\subsubsection{Terminolog\'ia}

A continuación se incluye un listado con terminolog\'ia relevante que se usara en la 
especificaci\'on de los protocolos.

\subsubsubsection{Recurso}

Objeto o servicio de la red que puede ser identificado por una URI

\subsubsubsection{Cliente}

Un programa que establece la conexi\'on con el objetico de enviar pedidos.

\subsubsubsection{User Agent}

El cliente que inicializa un pedido. Estos suelen ser navegadores, editores u otro 
tipo de herramientas usadas por un usuario.

\subsubsubsection{Servidor}

Una aplicaci\'on que acepta conexiones para proveer respuestas y servicios en base 
a pedidos. Un programa puede ser tanto como cliente como servidor, estos terminos 
hacen \'unicamente referencia al rol que el programa esta ocupando.

\subsubsubsection{Servidor de origen}

El servidor en donde un recurso en particular reside o va a ser creado.

\subsubsubsection{Proxy}

Un proxy es un programa intermediario que actua tanto como cliente como servidor 
con el objetivo de hacer pedidos de parte de otros clientes. Los pedidos son procesados internamente 
o pasandolos, posiblemente habiendo hecho ciertos cambios, a otros servidores. Un 
"proxy transparente" es un proxy que no modifica ni el pedido ni la respuesta exceptuando autenticaci\'on e 
identificaci\'on. Un "proxy no transparente" es un proxy que modifica el pedido o la respuesta con el 
objetivo de agregarle cierto servicio al user agent.

Para el objetivo de los protocolos implementados el concepto de proxy es escencial dado que la aplicaci\'on 
a implementarse (y que usara dichos protocolos) es un proxy.

% Otros terminos pueden ser incluidos en esta sección!!!

\newpage
\subsection{Protocolos}
\subsubsection{Protocolo para acceder al monitoreo del proxy}

\subsubsubsection{Funcionamiento general}

Este protocolo es un protocolo de pedido y respuesta con el objetivo de obtener informaci\'on 
de monitero del proxy. Para poder realizar la comunicaci\'on correctamente se env\'ian datos 
de autenticaci\'on en el pedido del cliente. Para que el servidor realmente devuelva la 
informaci\'on correcta, es importante que la autenticaci\'on sea correcta. El formato y la 
informaci\'on enviada tanto en el pedido como en la respuesta ser\'an aclarados en secciones 
posteriores.

Es necesario precisar que toda conexi\'on dada por este protocolo es iniciada por un user 
agent y en este caso un proxy funciona de servidor para proveer a un usuario datos 
posiblemente relevantes del mismo.

Para poder utilizar este protocolo se reservar\'a un puerto especifico que debe ser conocido 
por el cliente para poder establecer la conexi\'on. No es exclusivo, pero se asume que este 
protocolo es montado sobre TCP/IP, corroborando as\'i integridad y seguridad en la transferencia 
de los datos.

Por \'ultimo es util resaltar que en cada pedido de este protocolo se esta generando una nueva 
conexi\'on.
%Hasta este punto realmente llega la subsubsubsección anterior
%ALVAMASTER CUCHAME, NO HAY QUE PONERLE NINGUN TIPO DE HEADER ACLARANDO QUE PROTOCOLO ES?!?!?!

Puerto X reservado para acceder al monitoreo del proxy.

User: username.
Pass: password.

En caso de autenticarse correctamente, el usuario podr\'a acceder a la información de monitoreo.

Bytes-transferred: cantidad total de bytes transferidos.
Bytes-transferred-clients: bytes transferidos entre clientes y el proxy.
Bytes-transferred-servers: bytes transferidos entre el proxy y los origin servers.
Total-filters-applied:
All-access:
Ip-filter:
URL-filter:
Media-type-filter:
Size-filter:
L33t-transformations:
Images-transformations:
Connections-open: cantidad de conexiones abiertas (clientes y origin server)


200 OK

401 UNAUTHORIZED

\newpage
\subsubsection{Protocolo para acceder a la configuración del proxy}

\subsubsubsection{Funcionamiento general}

Este protocolo es un protocolo de pedido y respuesta con el objetivo modificar la 
configuraci\'on actual del proxy. Si bien es un protocolo de pedido y respuesta, la 
respuesta no provee informaci\'on particularmente relevante sino que simplemente es 
una confirmaci\'on del resultado de la operaci\'on. 

El formato y la informaci\'on enviada tanto en el pedido como en la respuesta ser\'an 
aclarados en secciones posteriores.

Es necesario precisar que toda conexi\'on dada por este protocolo es iniciada por un user 
agent y en este caso un proxy funciona de servidor, modificando su configuraci\'on y retornando 
un c\'odigo de respuesta.

Para poder utilizar este protocolo se reservar\'a un puerto especifico que debe ser conocido 
por el cliente para poder establecer la conexi\'on. No es exclusivo, pero se asume que este 
protocolo es montado sobre TCP/IP, corroborando as\'i integridad y seguridad en la transferencia 
de los datos.

Por \'ultimo es util resaltar que en cada pedido de este protocolo se esta generando una nueva 
conexi\'on.
%Hasta este punto realmente llega la subsubsubsección anterior
%ALVAMASTER CUCHAME, NO HAY QUE PONERLE NINGUN TIPO DE HEADER ACLARANDO QUE PROTOCOLO ES?!?!?!

Puerto Y reservado para acceder a la configutaci\'on del proxy.

config-filter 1 [block | unblock]
config-filter 2 [block | unblock] [ip-adress or mask]
config-filter 3 [block | unblock] url-regexp
config-filter 4 [block | unblock] media-type
config-filter 5 [block | unblock] size-in-bytes
config-filter 6 [enable | disable]
config-filter 7 [enable | disable]


200 OK

400 BAD REQUEST

500 INTERNAL ERROR

\newpage
\section{Problemas y dificultades encontrados}

concurrencia
selector 1 thread  varios thread --> sincronizacion hiper heavy!

\newpage
\section{Casos de prueba}



\end{document}
