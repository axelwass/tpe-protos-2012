\documentclass[a4paper,10pt]{article}

\usepackage[utf8]{inputenc}
\usepackage{t1enc}

\usepackage[utf8]{inputenc}
\usepackage{t1enc}
\usepackage[spanish]{babel}
\usepackage[pdftex,usenames,dvipsnames]{color}
\usepackage[pdftex]{graphicx}
\usepackage{enumerate}
\usepackage{amsmath}
\usepackage{amsfonts}
\usepackage{amssymb}
\usepackage[table]{xcolor}
\usepackage[small,bf]{caption}
\usepackage{float}
\usepackage{subfig}
\usepackage{listings}
\usepackage{bm}
\usepackage{times}

\begin{document}
\setcounter{secnumdepth}{5}
\setcounter{tocdepth}{5}

\renewcommand{\lstlistingname}{C\'odigo Fuente}
\lstloadlanguages{Octave} 
\lstdefinelanguage{MyOctave}[]{Octave}{
        deletekeywords={beta,det},
        morekeywords={repmat}
} 
\lstset{
        language=MyOctave,
        stringstyle=\ttfamily,
        showstringspaces = false,
        basicstyle=\footnotesize\ttfamily,
        commentstyle=\color{gray},
        keywordstyle=\bfseries,
        numbers=left,
        numberstyle=\ttfamily\footnotesize,
        stepnumber=1,                   
        framexleftmargin=0.20cm,
        numbersep=0.37cm,              
        backgroundcolor=\color{white},
        showspaces=false,
        showtabs=false,
        frame=l,
        tabsize=4,
        captionpos=b,               
        breaklines=true,             
        breakatwhitespace=false,      
        mathescape=true
}

%%%%%%%%%%%%%%%%%%%%%%%%%%%%%%%%%%
%%%%%%%% begin TITLE PAGE %%%%%%%%
%%%%%%%%%%%%%%%%%%%%%%%%%%%%%%%%%%
\begin{titlepage}
        \vfill
        \thispagestyle{empty}
        \begin{center}
                \includegraphics{./images/itba_logo.png}
                \vfill
                \Huge{Trabajo Pr\'actico Especial - Pre-entrega}\\
                \vspace{1cm}
                \Huge{Protocolos de Comunicaci\'on}\\
        \end{center}
        \vfill
        \large{
        \begin{tabular}{lcr}
                Crespo, Alvaro && 50758 \\
                Susnisky, Dar\'io && 50592\\
                Wassington, Axel && 50124\\
        \end{tabular}
}
        \vspace{2cm}
        \begin{center}
                \large{9 de mayo de 2012}\\
        \end{center}
\end{titlepage}
\newpage

\setcounter{page}{1}

\tableofcontents

\newpage
\section{RFCs relevantes para el desarrollo de este trabajo pr\'actico}

\begin{itemize}
 \item rfc2616 --- http 1.1
 \item rfc2119 --- keywords to indicate requirement levels
 \item rfc822 ---- Backus-Naur Form (BNF)
 \item rfc2047 --- Message Header Extensions non ASCII-text (???)
 \item rfc2396 --- Uniform Resource Identifiers (URI): Generic Syntax

 \item rfcs 1950,51,52 --- gzip enconding and deflate enconding
 \item rfc1766 --- Tags for the Identification of Languages
 \item rfc1738 ---Uniform Resource Locators (URL)
\end{itemize}


\newpage
\section{Protocolos desarrollados}

    \subsection{Introducci\'on}

    A continuación se presentar las especificaciones para dar un breve entendimiento 
    de los protocolos para acceder al monitoreo del proxy y para acceder a la configuraci\'on del mismo.

        \subsubsection{Proposito}

        Tanto el protocolo para acceder al monitoreo del proxy como el protocolo para acceder a la 
        configuración de este son protocolos que funcionan a nivel aplicación para sistemas de informaci\'on.

        El protocolo para configurar al proxy pretende enviar los datos crudos para lograr su objetivo. En cambio, 
        el protocolo para el monitero requiere un nivel de autenticaci\'on para luego recibir la informaci\'on 
        deseada. 

        Ninguno de los dos protocolos posee configuración extendida dando uso especifico tanto para el monitoreo 
        como para la configuración particular de un proxy.

        \subsubsection{Requerimientos}

        Las palabras DEBE, NO DEBE,REQUERIDO, HARA, NO HARA, DEBERIA, NO DEBERIA, 
        RECOMENDADO, PUEDE y OPCIONAL se deben interpretar tal y como describe el 
        RFC2119 (http://www.rfc-es.org/rfc/rfc2119-es.txt).

        \subsubsection{Terminolog\'ia}

        A continuación se incluye un listado con terminolog\'ia relevante que se usara en la 
        especificaci\'on de los protocolos.

            \paragraph*{Recurso}

            Objeto o servicio de la red que puede ser identificado por una URI

            \paragraph*{Cliente}

            Un programa que establece la conexi\'on con el objetico de enviar pedidos.

            \paragraph*{User Agent}

            El cliente que inicializa un pedido. Estos suelen ser navegadores, editores u otro 
            tipo de herramientas usadas por un usuario.

            \paragraph*{Servidor}

            Una aplicaci\'on que acepta conexiones para proveer respuestas y servicios en base 
            a pedidos. Un programa puede ser tanto como cliente como servidor, estos terminos 
            hacen \'unicamente referencia al rol que el programa esta ocupando.

            \paragraph*{Servidor de origen}

            El servidor en donde un recurso en particular reside o va a ser creado.

            \paragraph*{Proxy}

            Un proxy es un programa intermediario que actua tanto como cliente como servidor 
            con el objetivo de hacer pedidos de parte de otros clientes. Los pedidos son procesados internamente 
            o pasandolos, posiblemente habiendo hecho ciertos cambios, a otros servidores. Un 
            "proxy transparente" es un proxy que no modifica ni el pedido ni la respuesta exceptuando autenticaci\'on e 
            identificaci\'on. Un "proxy no transparente" es un proxy que modifica el pedido o la respuesta con el 
            objetivo de agregarle cierto servicio al user agent.

            Para el objetivo de los protocolos implementados el concepto de proxy es escencial dado que la aplicaci\'on 
            a implementarse (y que usara dichos protocolos) es un proxy.

            % Otros terminos pueden ser incluidos en esta sección!!!



    \subsection{Convenciones de notaci\'on y gram\'atica}

        \subsubsection{BNF aumentada}

        Todos los mecanismos especificados en este documento son descriptos tanto en prosa como en Backus-Naur Form (BNF) aumentada, similar a la usada por el RFC 822.
        \label{RFC 822}
        Los implementadores deber\'an estar familiarizados con esta notaci\'on para enteder esta especificaci\'on. La BNF aumentada incluye las siguientes estructuras:

        \begin{itemize}
            \item nombre = definici\'on\\
                El nombre de una regla es simplemente el nombre por s\'i mismo (sin estar contenidos por "<" y ">") y está separada de su definici\'on por el caracter igual "=".
                El espacio en blanco es solo significante en que la indentaci\'on de l\'ineas continuadas es usada para indicar una definici\'on de una regla que abarca m\'as 
                de una l\'inea. Algunas reglas b\'asicas estas en may\'usculas como SP, LWS, HT, CRLF, DIGIT, ALPHA, etc. Los s\'imbolos "<" y ">" son usados en las definiciones en los casos
                en que su presencia facilite el entendimiento del uso de los nombres de regla.
            \item "literal" \\
                Las comillas doblas rodean texto literal. A menos que expl\'icitamente se exprese lo contrario, dicho texto es \textit{case-insensitive}
            \item regla1 | regla2\\
                Los elementos separados por una barra vertical ("|") son alternativos, por ejemplo "si | no" aceptar\'a si o no.
            \item (regla1 regla2)\\
                Los elementos encerrados por par\'entesis son tratados como un solo elemento. Luego $"(elem (foo | bar) elem)"$ permite las secuencias 
                $"elem\,foo\,elem"$ and $"elem\,bar\,elem"$.
            \item *regla\\
                El caracter "*" precediendo a un elemento indica repetici\'on. La forma completa es  "<n>*<m>elemento" indicando que puede haber al menos <n> y como mucho <m>
                ocurrencias del elemento. Los valores \textit{default} son $0$ e infinito, asique "*(elemento)" permite cualquier n\'umero, incluyendo cero; "1*elemento" 
                requiere al menos uno; y "1*2element" permite uno o dos.
            \item \verb+[+ regla \verb+]+\\
                Los corchetes encierran elementos opcionales; "[foo bar]" es equivalente a "*1(foo bar)".
            \item N regla\\
                Repetici\'on espec\'ifica: "<n>(elemento)" es equivalente a "<n>*<n>(elemento)";esto es, exactamente <n> ocurrencias de (element). Luego, 2DIGIT es un n\'umero 
                de 2 cifras, y 3ALPHA es una cadena de 3 caracteres alfab\'eticos.
            \item \#regla\\
                Se utiliza para definir listas de elementos. La forma completa es "<n>\#<m>elemento" indicando por lo menos <n> y como mucho <m> elementos, cada uno
                separado por una o m\'as comas (",") y OPCIONALMENTE espacio linear en blanco, \textit{linear white space}, (LWS). Esto hace que la forma normal de usar listas 
                sea muy sencilla; una regla como\\
                \[( \text{*}LWS\,elemento\,\text{*}(\,\text{*}LWS\,","\,\text{*}LWS\,elemento ))\]
                puede representarse como\\
                \[1\#elemento\]
                Cuando esta estructura sea usada, los elementos nulos, $null$, son permitidos, pero no contribuyen a la cantidad de elementos presentes. Esto es, 
                $"(elemento), , (elemento) "$  est\'a permitido, pero solo cuenta como 2 elementos. Por lo tanto, cuando al menos uno elemento sea requerido, al menos un elemento
                no nulo DEBE estar presente. Los valores \textit{default} son $0$ e infinito, asique "\#element" permite cualquier cantidad, incluyen cero; $"1\#elemento"$ requiere
                al menos uno; y $"1\#2elemento"$ permite uno o dos.
            \item ; comentario\\
                Un punto y coma, puesto a alguna distancia a la derecha de una regla de texto, empieza un comentario que continua hasta el final de l\'inea. Esta es una forma muy
                simple de incluir notas \'utiles en media de las especificaciones.
            \item *LWS implicado\\
                La gram\'atica descripta por esta especificaci\'on es basada en palabras. Excepto cuando se exprese lo contrario, el espacio linear en blanco (LWS) puede ser 
                incluido entre dos palabras adyacentes, y entre palabras y separadores adyacentes, sin cambiar la interpretaci\'on del campo. Al menos un delimitador 
                (LWS y/o separadores) DEBE existir entre dos \textit{tokens} cualesquiera, (para la definici\'on de "token" que sigue), dado que sino ser\'ian interpretados 
                como un solo \textit{token}.        
        \end{itemize}

        \subsubsection{Reglas b\'asicas}

        Las siguientes reglas son usadas a lo largo de esta especificaci\'on para describir estructuras b\'asicas de an\'alisis sint\'actico.
        El set de caracteres US-ASCII est\'a definido por ANSI X3.4-1986.
        
        \begin{eqnarray*} 
            OCTET          & = & <cualquier\,secuencia\,de\,datos\,de\,8\,bits> \nonumber\\
            CHAR           & = & <cualquier\,caracter\,US-ASCII\,(octetos\,0\,-\,127)> \nonumber\\
            UPALPHA        & = & <cualquier\,letra\,may\'uscula\,US-ASCII\,"A".."Z"> \nonumber\\
            LOALPHA        & = & <cualquier\,letra\,min\'uscula\,US-ASCII"a".."z">  \nonumber\\
            ALPHA          & = & UPALPHA\,|\,LOALPHA  \nonumber\\
            DIGIT          & = & <cualquier\,d\'igito\,US-ASCII\,"0".."9">  \nonumber\\
            CTL            & = & <cualquier\,caracter\,de\,control\,US-ASCII\,(octetos\,0\,-\,31)\,y\,DEL\,(127)>  \nonumber\\
            CR             & = & <retorno\,de\,carro,US-ASCII\,CR,\,(13)>  \nonumber\\
            LF             & = & <salto\,de\,l\'inea,US-ASCII\,LF,\,(10)>  \nonumber\\
            SP             & = & <espacio,US-ASCII\,SP,\,(32)>  \nonumber\\
            HT             & = & <tab\,horizontal,US-ASCII\,HT,\,(9)>  \nonumber\\
            <">            & = & <comillas\,dobles,US-ASCII,\,(34)>  \nonumber\\
        \end{eqnarray*}

        Nuestra especificaci\'on define la secuencia CR LF como el marcador de fin de l\'inea para todos los elementos del protocolo. 

        \begin{eqnarray*}
            CRLF         &  =  & CR LF \nonumber\\
        \end{eqnarray*}

        Todo espacio linear en blanco (LWS) tiene la misma sem\'antica que SP. Un recipiente PUEDE reemplazar cualquier LWS por un solo SP antes de interpretar el valor del 
        campo o pasar el mensaje

        \begin{eqnarray*}
                LWS          & = & [CRLF] 1\text{*}( SP | HT ) \nonumber\\
        \end{eqnarray*}

        La regla TEXT se usa solamente para contenidos descriptivos de un campo y valores que no se supone que sean interpretados por el \textit{parser} del mensaje.
        Palabras de  *TEXT PUEDEN contener caracteres de otros sets de caracteres aparte del ISO- 8859-1, solo cuando sean encodeados de acuerdo a las reglas del 
        RFC 2047 \label{RFC 2047}

        \begin{eqnarray*}
                TEXT         & = & <cualquier\,OCTET\,excepto\,CTLs,\nonumber\\
                            &   & pero\,incluyendo\,LWS>\nonumber\\
        \end{eqnarray*}

        Los caracteres num\'ericos hexadecimal son usados en varios elementos del protocolo.

        \begin{eqnarray*}
                HEX           & = &   ``A"\,|\,``B"\,|\,``C"\,|\,``D"\,|\,``E"\,|\,``F"\nonumber\\
                            &   & \,|\,``a"\,|\,``b"\,|\,``c"\,|\,``d"\,|\,``e"\,|\,``f"\,|\,DIGIT\nonumber\\
        \end{eqnarray*}


        Muchos valores de campos consisten en palabras separadas por un LWS o caracteres especiales. Estos caracteres especiales DEBEN estar entre commillas dobles 
        para ser usados dentro de un valor de un par\'ametro.

        \begin{eqnarray*}
            token         & = & 1\text{*}<cualquier\,CHAR\,excepto\,CTLs\,o\,separadores>\nonumber\\
            separadores    & = & ``("\,\,|\,\,``)"\,\,|\,\,``<"\,\,|\,\,``>"\,\,|\,\,``@"\nonumber\\
                &   &\,\,|\,\,``,"\,\,|\,\,``;"\,\,|\,\,``:"\,\,|\,\,``\backslash"\,\,|\,\,<"\,''\,``>\nonumber\\
                &   &\,\,|\,\,``/"\,\,|\,\,``["\,\,|\,\,``]"\,\,|\,\,``?"\,\,|\,\,``="\nonumber\\
                &   &\,\,|\,\,``\verb+{+"\,\,|\,\,``\verb+}+"\,\,|\,\,SP\,\,|\,\,HT\nonumber\\
        \end{eqnarray*}


        Se puede incluir comentarios en algunos campos rodeando el texto de comentario con par\'entesis. Los comentarios son permitidos solamente en aquellos campos 
        que contengan "comentario" como parte de su definici\'on de valor de campo. En todos los otros casos, los par\'entesis son considerados como parte del valor del campo.
        \begin{eqnarray*}
            comentario       & = & "("\,\,\text{*}(\,ctext\,\,|\,\,|\,\,comentario,)\,\,")"\nonumber\\
            ctext         & = & <cualquier\,\,TEXT\,\,excluyendo\,\,``("\,\,y\,\,``)">\nonumber\\
        \end{eqnarray*}



        Una cadena de texto es analizada como una sola palabra se esta rodeada de comillas dobles.

        \begin{eqnarray*}
            quoted-string & = & (\,\,<">\,\,\text{*}(qdtext\,\,|\,\,)\,\,<">\,\,)\nonumber\\
            qdtext        & = & <cualquier\,\,TEXT\,\,excepto\,\,<">>\nonumber\\
        \end{eqnarray*}

\newpage

    \subsection{Protocolo para acceder al monitoreo del proxy}

        \paragraph*{Funcionamiento general}

        Este protocolo es un protocolo de pedido y respuesta con el objetivo de obtener informaci\'on 
        de monitero del proxy. Para poder realizar la comunicaci\'on correctamente se env\'ian datos 
        de autenticaci\'on en el pedido del cliente. Para que el servidor realmente devuelva la 
        informaci\'on correcta, es importante que la autenticaci\'on sea correcta. El formato y la 
        informaci\'on enviada tanto en el pedido como en la respuesta ser\'an aclarados en secciones 
        posteriores.

        Es necesario precisar que toda conexi\'on dada por este protocolo es iniciada por un user 
        agent y en este caso un proxy funciona de servidor para proveer a un usuario datos 
        posiblemente relevantes del mismo.

        Para poder utilizar este protocolo se reservar\'a un puerto especifico que debe ser conocido 
        por el cliente para poder establecer la conexi\'on. No es exclusivo, pero se asume que este 
        protocolo es montado sobre TCP/IP, corroborando as\'i integridad y seguridad en la transferencia 
        de los datos.

        Por \'ultimo es util resaltar que en cada pedido de este protocolo se esta generando una nueva 
        conexi\'on.
        %Hasta este punto realmente llega la subsubsubsección anterior
        %ALVAMASTER CUCHAME, NO HAY QUE PONERLE NINGUN TIPO DE HEADER ACLARANDO QUE PROTOCOLO ES?!?!?!

        \subsubsection{Parametros del Protocolo}

        \subsubsection{Mensaje del Protocolo}
            Los mensajes del protocolo consisten de solicitudes por parte del cliente y respuestas por parte del servidor proxy.

            \begin{eqnarray*}
                mensaje  & = &\,\,Solicitud\,\,|\,\,Respuesta\nonumber\\
            \end{eqnarray*}

        \subsubsection{Solictud}
        Una solicitud del cliente al servidor proxy incluye el username y el password del usuario que actual.

        \begin{eqnarray*}
                Solicitud\ & = & \verb+"monitor"+\,\,SP\,\,User\,\,SP\,\,Password\nonumber\\
        \end{eqnarray*}


        \paragraph*{User y Password}
        User, el nombre del usuario, y password, su contraseña deben ser proveidos para iniciar la comunicaci\'on
        con este protocolo. El servidor autenticara estos datos para permiter o rechazar el monitereo.
            
    \subsubsection{Respuesta}
    
    Los mensajes de respuesta tienen como contenido en primer lugar el c\'odigo de respuesta y en caso de ser un 
    pedido aceptado tambi\'en contendra la informaci\'on correspondiente al monitereo.

    \begin{eqnarray*}
            Respuesta\ & = &\,\,Codigo\,\,SP\,\,\verb+[Datos de monitereo]+\nonumber\\
    \end{eqnarray*}
    
    \paragraph*{C\'odigo de respuesta}
        El c\'odigo de respuesta indica si la operaci\'on fue realizada con \'exito o no.

        \begin{eqnarray*}
            Codigo\ & = & \verb+"200 OK"+\,\,|\,\,\verb+"401 UNAUTHORIZED"+\nonumber\\
        \end{eqnarray*}

    \subsubsubsection{Datos de monitereo}

    El protocolo tiene el potencial de proveer al usuario de ciertos datos del monitoreo del proxy que se 
    especifican a continuaci\'on

    \paragraph*{Bytes-transferred:} Cantidad total de bytes transferidos.
    \paragraph*{Bytes-transferred-clients:} Bytes transferidos entre clientes y el proxy.
    \paragraph*{Bytes-transferred-servers:} Bytes transferidos entre el proxy y los origin servers.
    \paragraph*{Total-filters-applied:} Cantidad de filtros que fueron aplicados.
    \paragraph*{All-access:} Cantidad de pedidos.
    \paragraph*{Ip-filter:} Cantidad de pedidos filtrados por la direcci\'on IP.
    \paragraph*{URL-filter:} Cantidad de pedidos filtrados por la URL.
    \paragraph*{Media-type-filter:} Cantidad de pedidos filtrados por el tipo de recurso.
    \paragraph*{Size-filter:} Cantidad de pedidos filtrados por tamaño.
    \paragraph*{L33t-transformations:} Cantidad de conversiones a lenguaje L33t.
    \paragraph*{Images-transformations:} Cantidad de imagenes transformadas.
    \paragraph*{Connections-open:} Cantidad de conexiones abiertas (clientes y origin server)

\newpage
    \subsection{Protocolo para acceder a la configuración del proxy}

        \paragraph*{Funcionamiento general}

        Este protocolo es un protocolo de pedido y respuesta con el objetivo modificar la 
        configuraci\'on actual del proxy. Si bien es un protocolo de pedido y respuesta, la 
        respuesta no provee informaci\'on particularmente relevante sino que simplemente es 
        una confirmaci\'on del resultado de la operaci\'on. 

        El formato y la informaci\'on enviada tanto en el pedido como en la respuesta ser\'an 
        aclarados en secciones posteriores.

        Es necesario precisar que toda conexi\'on dada por este protocolo es iniciada por un user 
        agent y en este caso un proxy funciona de servidor, modificando su configuraci\'on y retornando 
        un c\'odigo de respuesta.

        Para poder utilizar este protocolo se reservar\'a un puerto especifico que debe ser conocido 
        por el cliente para poder establecer la conexi\'on. No es exclusivo, pero se asume que este 
        protocolo es montado sobre TCP/IP, corroborando as\'i integridad y seguridad en la transferencia 
        de los datos.

        Por \'ultimo es util resaltar que en cada pedido de este protocolo se esta generando una nueva 
        conexi\'on.
        %Hasta este punto realmente llega la subsubsubsección anterior
        %ALVAMASTER CUCHAME, NO HAY QUE PONERLE NINGUN TIPO DE HEADER ACLARANDO QUE PROTOCOLO ES?!?!?!

    Puerto Y reservado para acceder a la configutaci\'on del proxy.

        \subsubsection{Parametros de los protocolos}


        \subsubsection{Mensaje del protocolo}

            \paragraph*{Tipos de mensaje}
            Los mensajes del protocolo consisten de solicitudes por parte del cliente y respuestas por parte del servidor proxy.

            \begin{eqnarray*}
                mensaje  & = &\,\,Solicitud\,\,|\,\,Respuesta\nonumber\\
            \end{eqnarray*}


        \subsubsection{Solictud}
        Una solicitud del cliente al servidor proxy incluye el identificador del filtro o transformador, la operaci\'on deseada y uno o m\'as par\'ametros opcionales.

%         config-filter n (operaci\'on) [parametros]
        \begin{eqnarray*}
                Solicitud\ & = & \verb+"config-filter"+\,\,SP\,\,id-filtro\,\,SP\,\,operacion\,\,SP\,\,\verb+[+parametros\verb+]+\,\,CRLF\nonumber\\
        \end{eqnarray*}

            \paragraph*{Identificadores de filtros}
            Los identificadores de los filtros son simplemente n\'umeros decimales. Hasta el momento hay 7 filtros, en caso de agregarse uno nuevo se deber\'a 
            asignar el n\'umero correspondiente como su identificador.

            \begin{eqnarray*}
                id-filtro & = & 1\,\,|\,\,2\,\,|\,\,3\,\,|\,\,4\,\,|\,\,5\,\,|\,\,6\,\,|\,\,7\nonumber\\
            \end{eqnarray*}
            
            \paragraph*{Operaci\'on}
            La operaci\'on a realizar indica la habilitaci\'on, o no, del filtro. Por lo que solo hay dos operaciones posibles, habilitar o deshabilitar.

            \begin{eqnarray*}
                operacion & = & (\verb+"enable"+\,\,|\,\,\verb+"disable"+)\nonumber\\
            \end{eqnarray*}
            
            \paragraph*{Par\'metros}
            Algunos m\'etodos requieren de uno o m\'as par\'ametro. Dada que estos varian de manera significante de acuerdo al m\'etodo, se los discutir\'a 
            posteriormente.
        
        \subsubsection{Respuesta}

        200 OK

        400 BAD REQUEST

        500 INTERNAL ERROR


        \subsubsection{M\'etodos}
        
            \paragraph*{Filtro Nº1}

            Este m\'etodo no requiere par\'ametros.
%             config-filter 1 (enable | disable)

            \paragraph*{Filtro Nº2}
             config-filter 2 (enable | disable) [ip-adress or mask]

            \paragraph*{Filtro Nº3}
            config-filter 3 (enable | disable) url-regexp

            \paragraph*{Filtro Nº4}
            config-filter 4 (enable | disable) media-type

            \paragraph*{Filtro Nº5}
            config-filter 5 (enable | disable) size-in-bytes

            \paragraph*{Filtro Nº6}
            config-filter 6 (enable | disable)

            \paragraph*{Filtro Nº7}
            config-filter 7 (enable | disable)
\newpage
\section{Problemas y dificultades encontrados}

selector 1 thread  o varios thread --> sincronizacion hiper heavy!

mantener estado interno de cada conexion

\newpage
\section{Casos de prueba}



\end{document}
